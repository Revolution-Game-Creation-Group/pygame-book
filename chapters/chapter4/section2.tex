\section{Τα Βλέπω Διπλά! Δύο Bouncing Balls}

Πως θα μετατρέψετε το bouncing ball ώστε να έχει δύο μπάλες; Ελπίζουμε να μη σας φαίνεται δύσκολο. Ας δούμε μια απλοϊκή λύση. Κάτω από την γραμμή:

\begin{minted}[bgcolor=bg, frame=lines, framesep=10pt]{python}
xspeed,yspeed = 50.0 , 50.0
\end{minted}

προσθέστε:

\begin{minted}[bgcolor=bg, frame=lines, framesep=10pt]{python}
x2,y2 = 50.0, 50.0
xspeed2,yspeed2= 30,30
\end{minted}

Βάζουμε λίγο διαφορετική ταχύτητα για να καταλαβαίνουμε ποια μπάλα είναι η
πρώτη και ποια η δεύτερη. Επίσης ξεκινάμε και από άλλη αρχική θέση. Δίνουμε τώρα όλο τον κύριο βρόχο του προγράμματος (αφαιρέσαμε τα αρχικά σχόλια για συντομία):

\begin{minted}[bgcolor=bg, linenos,frame=lines, framesep=10pt]{python}
  endprogram = False
  while not endprogram:
    screen.fill(surfacecolor)
    screen.blit(ball, (x, y))
    screen.blit(ball, (x2,y2))
    time = clock.tick()
    thetext = textfont.render(str(1000/time), True, (255,0,0),(255,255,0))
    screen.blit(thetext,(0,0))
    time = time / 1000.0
    distance_x = time * xspeed
    distance_y = time * yspeed
    distance_x2 = time * xspeed2
    distance_y2 = time * yspeed2
    x = x + distance_x
    y = y + distance_y
    x2 = x2 + distance_x2
    y2 = y2 + distance_y2
    if (x > (640.0-ballwidth) or x<=0.0):
      xspeed = -xspeed
    if (y > (480.0-ballheight) or y<=0.0):
      yspeed = -yspeed
\end{minted}

\begin{minted}[bgcolor=bg, firstnumber=22, linenos,frame=lines, framesep=10pt]{python}
    if (x2 > (640.0-ballwidth) or x2<=0.0):
      xspeed2 = -xspeed2
    if (y2 > (480.0-ballheight) or y2<=0.0):
      yspeed2 = -yspeed2

    pygame.display.update()
    endprogram = getQuit()

  pygame.quit()
  exit()
\end{minted}

Είναι εύκολο, αλλά είναι και κακογραμμένο! Γιατί, αν το να βάλουμε
μερικές ακόμα μεταβλητές για να φτιάξουμε μια δεύτερη μπάλα είναι απλό, τι
θα λέγατε αν σας έλεγα να το φτιάξετε για 20 μπάλες; ή για 100 μπάλες;
Θα πρέπει σίγουρα να αλλάξετε τρόπο σκέψης! Και φυσικά ο αντικειμενοστραφής
προγραμματισμός (που θα δούμε στο επόμενο κεφάλαιο) θα μας βοηθήσει.

Α, και τώρα φυσικά που θα το τρέξετε, θα δείτε και ποια μπάλα περνάει πάνω
από την άλλη. Και το αποτέλεσμα είναι απόλυτα λογικό.

Κατεβάστε τα προγράμματα αυτού του κεφαλαίου: Hello Pygame: \url{http://www.freebsdworld.gr/files/hello-pygame.zip}, Colorbars: \url{http://www.freebsdworld.gr/files/colorbars.zip}, Bouncing Ball: \url{http://www.freebsdworld.gr/files/bouncing.zip}
