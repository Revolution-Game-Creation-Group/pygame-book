\chapter {Νυχτερινή Περιπέτεια!}
\label{chap:adventure-game}
\section{Εισαγωγή στις Λίστες}
Πριν ξεκινήσουμε να γράφουμε το επόμενο μας παιχνίδι, πρέπει να μιλήσουμε λίγο
για μια σημαντική δομή δεδομένων της python: Τις {\em λίστες}. Βλέπετε οι απλές
μεταβλητές από μόνες τους δεν είναι επαρκείς για να χειριστούμε πλήθος
δεδομένων. Το σημαντικότερο μειονέκτημα τους είναι ότι δεν μπορούμε να
επεξεργαστούμε μαζικά και με τον ίδιο τρόπο απλές μεταβλητές που περιέχουν
δεδομένα στα οποία θέλουμε να εκτελέσουμε την ίδια διαδικασία. Π.χ. να έχουμε
μια σειρά από τιμές προϊόντων τις οποίες θέλουμε όλες να αυξήσουμε κατά 10\%.

Αν το κάνουμε αυτό με απλές μεταβλητές, θα πρέπει να έχουμε μια μεταβλητή ανά
προϊόν και να γράψουμε τον ίδιο κώδικα όσες φορές είναι και τα προϊόντα.
Προφανώς αυτό δεν εξυπηρετεί.

Σε πολλές γλώσσες προγραμματισμού, βασική δομή για να επιτύχουμε μαζική
επεξεργασία δεδομένων είναι ο πίνακας. Η python ωστόσο μας παρέχει άλλες
δομές και μια από τις σημαντικότερες είναι η λίστα.

Η λίστα ανήκει σε ένα είδος δομής δεδομένων που ονομάζεται
{\em sequence ή ακολουθία} αν το προτιμάτε. Είναι εύκολο να φτιάξετε
μια λίστα:

\begin {minted}[bgcolor=bg, frame=lines, framesep=10pt]{python}
shoppinglist = [ "Cheese" , "Rice", "Coffee", "Milk", "Camba" ]
\end{minted}

Και εξίσου εύκολο να διατρέψετε όλα τα στοιχεία της:

\begin {minted}[bgcolor=bg, frame=lines, framesep=10pt]{python}
for element in shoppinglist:
  print element
\end{minted}

Είναι ακόμα αστεία εύκολο να δούμε αν κάτι ανήκει σε μια λίστα ή όχι:

\begin {minted}[bgcolor=bg, frame=lines, framesep=10pt]{python}
shoppinglist = [ "Cheese", "Rice", "Coffee", "Milk", "Camba" ]
element = raw_input("What are you buying today? ")
if element in shoppinglist:
  print "Yes, this is on the list"
else:
  print "Not on the list, you don't need it"
\end{minted}

Αν έχετε συνηθίσει σε άλλες γλώσσες να χρησιμοποιείτε την {\em for} για να
δημιουργείτε βρόχο με μετρητή, στην python θα το κάνετε ως εξής:

\begin {minted}[bgcolor=bg, frame=lines, framesep=10pt]{python}
for i in [1,2,3,4,5,6,7,8,9,10]:
  print i
\end{minted}

Αυτό δεν βολεύει αν θέλουμε να μετρήσουμε μέχρι το 1000, οπότε θα γράφαμε:

\begin {minted}[bgcolor=bg, frame=lines, framesep=10pt]{python}
for i in range(1,1001):
  print i
\end{minted}

Καθώς καταλαβαίνετε, η συνάρτηση {\tt range} παράγει μια λίστα με τα στοιχεία από 1 ως 1000 (όχι ως το 1001 όπως νομίζετε διαβάζοντας το παράδειγμα). Καλό είναι να γνωρίζετε ότι μια λίστα στην python μπορεί να
 περιέχει οτιδήποτε: αριθμούς, αλφαριθμητικά, αντικείμενα (που θα δούμε αργότερα) ακόμα και άλλες λίστες! Εκτός
 από τις λίστες, η python παρέχει επίσης και τα {\em tuples}, τα οποία είναι
 παρόμοια όμως μετά τη δημιουργία τους δεν μπορούν να μεταβληθούν. Σε μια λίστα μπορούμε όπως θα δούμε σε επόμενα άρθρα να προσθέσουμε και να διαγράψουμε
 στοιχεία και να εκτελέσουμε μια σειρά από χρήσιμες λειτουργίες
 (π.χ. ταξινόμηση).

Είμαστε τώρα έτοιμοι να ξεκινήσουμε το επόμενο μας παιχνίδι: Την Νυχτερινή Περιπέτεια ή Adventure!
